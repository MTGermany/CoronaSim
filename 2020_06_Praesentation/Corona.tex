%#############################################
% New pdflatex format,
% only png, jpg (or svg?) images but positioning now ok
% template in ~/tex/inputs/template_folien.tex
% [TODO now specialized to Vkoem_Ma; make general]
%#############################################


%\documentclass[mathserif]{beamer}
\documentclass[mathserif,handout]{beamer}
%\usepackage{beamerthemeshadow}
\input{$HOME/tex/inputs/defsSkript}  %$
\input{$HOME/tex/inputs/styleBeamerCorona}   %$

\usepackage{graphicx}


%##############################################################

\begin{document}




%##############################################################
\frame{\frametitle{}
%##############################################################


\placebox[center]{0.50}{0.48}
 {\figSimple{0.95\textwidth}{figs/title.png}}

}





%##############################################################
\frame{\frametitle{Infektionsdynamik}{
%##############################################################

\vspace{-1.65em}

\fig{0.9\textwidth}{figs/coronaDynamicsTemplate.png}
}}


%##############################################################
\frame{\frametitle{Infektionsdynamik}{
%##############################################################

\vspace{-1.65em}

\fig{0.9\textwidth}{figs/coronaDynamicsR2-0.png}
}}


%##############################################################
\frame{\frametitle{Reproduktionszahl $R=2$}{
%##############################################################

\fig{0.9\textwidth}{figs/coronaDynamicsR2-1.png}

Im Verlauf einer Infektionsgeneration von z.B. $\tau=\unit[5]{Tage}$
steckt ein 
Infizierter zwei weitere an
}}


%##############################################################
\frame{\frametitle{Reproduktionszahl $R=2$}{
%##############################################################

\vspace{-0.65em}

\fig{0.9\textwidth}{figs/coronaDynamicsR2-2.png}

Exponentielle Kaskade: Infiziertenzahl $n=n_0 \exp(\ln(R) t/\tau)$
}}


%##############################################################
\frame{\frametitle{Reproduktionszahl $R=2$}{
%##############################################################

\vspace{-0.65em}

\fig{0.9\textwidth}{figs/coronaDynamicsR2-3.png}

Exponentielle Kaskade: Infiziertenzahl $n=n_0 \exp(\ln(R) t/\tau)$
}}


%##############################################################
\frame{\frametitle{Reproduktionszahl $R=2$}{
%##############################################################



\fig{0.9\textwidth}{figs/coronaDynamicsR2-4.png}

Exponentielle Kaskade: Infiziertenzahl $n=n_0 \exp(\ln(R) t/\tau)$

\pause \red{Was ist mit Immunit\"at?}
}}


%##############################################################
\frame{\frametitle{Reproduktionszahl $R=2$ mit Immunit\"at}{
%##############################################################

\vspace{-0.15em}

\fig{0.9\textwidth}{figs/coronaDynamicsR2s-1.png}

Am Anfang gleich: ein 
Infizierter steckt zwei weitere an, wird aber \blue{selbst immun}
}}


%##############################################################
\frame{\frametitle{Reproduktionszahl $R=2$ mit Immunit\"at}{
%##############################################################

\fig{0.9\textwidth}{figs/coronaDynamicsR2s-2.png}

Die exponentielle Kaskade startet wie im Fall ohne Immunit\"at, aber
die \blue{Durchseuchung (Zahl der Schutzschilde) nimmt zu}
}}


%##############################################################
\frame{\frametitle{Reproduktionszahl $R=2$ mit Immunit\"at}{
%##############################################################

\fig{0.9\textwidth}{figs/coronaDynamicsR2s-3.png}

Der b\"ose Virus trifft nun vermehrt auf immune Personen $\Rightarrow$
\blue{effektive $R$-Wert sinkt, der exponentielle Verlauf geht in S\"attigung}
}}


%##############################################################
\frame{\frametitle{Reproduktionszahl $R=2$ mit Immunit\"at}{
%##############################################################

\fig{0.9\textwidth}{figs/coronaDynamicsR2s-4.png}

Der Erreger findet immer weniger ansteckbare Personen
\blue{$\Rightarrow$ die Infektion trocknet aus, einige bleiben uninfiziert}
}}


%##############################################################
\frame{\frametitle{Interaktive Corona-Simulation
    \texttt{corona-simulation.de} }{
%##############################################################

\fig{0.7\textwidth}{figs/title.png}

Meine Corona-Simulation  gie\3t
dies und mehr in ein statistisches Modell
}}



\end{document}















