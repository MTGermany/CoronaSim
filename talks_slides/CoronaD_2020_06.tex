%#############################################
% New pdflatex format,
% only png, jpg (or svg?) images but positioning now ok
% template in ~/tex/inputs/template_folien.tex
% [TODO now specialized to Vkoem_Ma; make general]
%#############################################


\documentclass[mathserif]{beamer}
%\documentclass[mathserif,handout]{beamer}
%\usepackage{beamerthemeshadow}
\input{$HOME/tex/inputs/defsSkript}  %$
\input{$HOME/tex/inputs/styleBeamerCorona}   %$

\usepackage{graphicx}


%##############################################################

\begin{document}




%##############################################################
\frame{\frametitle{}
%##############################################################


\placebox[center]{0.50}{0.48}
 {\figSimple{0.95\textwidth}{figsCorona/coronaSim_2020-06.png}}

}





%##############################################################
\frame{\frametitle{Infektionsdynamik f\"ur eine Reproduktionszahl 
$R_0=2$ ohne Immunit\"at}
%##############################################################

\visible<1>{\placebox{0.5}{0.52}{
  \figSimple{0.90\textwidth}{figsCorona/coronaDynamicsTreeTemplate.jpg}}}
\visible<2>{\placebox{0.5}{0.52}{
  \figSimple{0.90\textwidth}{figsCorona/coronaDynamicsTree-0.jpg}}}
\visible<3>{\placebox{0.5}{0.52}{
  \figSimple{0.90\textwidth}{figsCorona/coronaDynamicsTree-1.jpg}}}
\visible<4>{\placebox{0.5}{0.52}{
  \figSimple{0.90\textwidth}{figsCorona/coronaDynamicsTree-2.jpg}}}
\visible<5>{\placebox{0.5}{0.52}{
  \figSimple{0.90\textwidth}{figsCorona/coronaDynamicsTree-3.jpg}}}
\visible<6>{\placebox{0.5}{0.52}{
  \figSimple{0.90\textwidth}{figsCorona/coronaDynamicsTree-4.jpg}}}

\visible<3>{\placebox{0.5}{0.10}{\parbox{0.9\textwidth}{
Im Verlauf einer Infektionsgeneration von z.B. $\tau=\unit[7]{Tage}$
steckt ein Infizierter zwei weitere an}}}

\visible<4->{\placebox{0.5}{0.10}{\parbox{0.9\textwidth}{
Exponentielle Kaskade: Infiziertenzahl $n=n_0 \exp(\ln(R) t/\tau)$}}}

}

%##############################################################
\frame{\frametitle{Infektionsdynamik f\"ur eine Reproduktionszahl 
$R_0=2$ mit Immunit\"at}
%##############################################################

\visible<1>{\placebox{0.5}{0.52}{
  \figSimple{0.90\textwidth}{figsCorona/coronaDynamicsTreeTemplate.jpg}}}
\visible<2>{\placebox{0.5}{0.52}{
  \figSimple{0.90\textwidth}{figsCorona/coronaDynamicsTreeImmune-0.jpg}}}
\visible<3>{\placebox{0.5}{0.52}{
  \figSimple{0.90\textwidth}{figsCorona/coronaDynamicsTreeImmune-1.jpg}}}
\visible<4>{\placebox{0.5}{0.52}{
  \figSimple{0.90\textwidth}{figsCorona/coronaDynamicsTreeImmune-2.jpg}}}
\visible<5>{\placebox{0.5}{0.52}{
  \figSimple{0.90\textwidth}{figsCorona/coronaDynamicsTreeImmune-3.jpg}}}
\visible<6>{\placebox{0.5}{0.52}{
  \figSimple{0.90\textwidth}{figsCorona/coronaDynamicsTreeImmune-4.jpg}}}

\visible<3>{\placebox{0.5}{0.10}{\parbox{0.9\textwidth}{
Am Anfang gleich: ein 
Infizierter steckt zwei weitere an, wird aber \blue{selbst immun}
}}}

\visible<4>{\placebox{0.5}{0.10}{\parbox{0.9\textwidth}{
Die exponentielle Kaskade startet wie im Fall ohne Immunit\"at, aber
die \blue{Durchseuchung (Zahl der Schutzschilde) nimmt zu}
}}}

\visible<5>{\placebox{0.5}{0.10}{\parbox{0.9\textwidth}{
Der b\"ose Virus trifft nun vermehrt auf immune Personen $\Rightarrow$
\blue{effektive $R$-Wert sinkt, der exponentielle Verlauf geht in S\"attigung}
}}}

\visible<6->{\placebox{0.5}{0.10}{\parbox{0.9\textwidth}{
Der Erreger findet immer weniger ansteckbare Personen
\blue{$\Rightarrow$ die Infektion trocknet aus, einige bleiben uninfiziert}
}}}



}


%##############################################################
\frame{\frametitle{Interaktive Corona-Simulation
    \texttt{corona-simulation.de} }{
%##############################################################

\fig{0.7\textwidth}{figsCorona/coronaSim_2020-06.png}

Meine Corona-Simulation  gie\3t
dies und mehr in ein statistisches Modell
}}



\end{document}















