


%#############################################
% New pdflatex format,
% only png, jpg (or svg?) images but positioning now ok
% template in ~/tex/inputs/template_folien.tex
% [TODO now specialized to Vkoem_Ma; make general]
%#############################################


\documentclass[mathserif,aspectratio=1610]{beamer}
%\documentclass[mathserif,handout]{beamer}
%\usepackage{beamerthemeshadow}
\input{$HOME/tex/inputs/defsSkript}  %$
\input{$HOME/tex/inputs/styleBeamerVkoekMa}   %$

\usepackage{graphicx}


%##############################################################

\begin{document}

\section*{Mutation}


%##############################################################
\frame{\frametitle{B117-Ausbreitung Deutschland}
%##############################################################

\visible<1>{
 \placebox{0.5}{0.45}{\figSimple{0.80\textwidth}{figsCorona/p117-1.png}}}

\visible<2>{
 \placebox{0.5}{0.45}{\figSimple{0.80\textwidth}{figsCorona/p117-2.png}}}

\visible<3-7>{
 \placebox{0.5}{0.45}{\figSimple{0.80\textwidth}{figsCorona/p117-3.png}}}

\visible<8>{
 \placebox{0.5}{0.45}{\figSimple{0.80\textwidth}{figsCorona/p117-3a.png}}}

\visible<9->{
 \placebox{0.5}{0.45}{\figSimple{0.80\textwidth}{figsCorona/p117-4.png}}}

\visible<4-8>{\placebox{0.57}{0.60}{\parbox{0.3\textwidth}{
\maindm{y=\frac{p}{1-p}}
}}}

\visible<5-8>{\placebox{0.785}{0.60}{\parbox{0.2\textwidth}{
\maindm{y=y_0e^{r_yt}}
}}}

\visible<6-8>{\placebox{0.85}{0.47}{\parbox{0.3\textwidth}{
\maindm{r_y \approx \frac{\ln y(t_2)-\ln y(t_1)}{t_2-t_1}}
}}}

\visible<7-8>{\placebox{0.785}{0.34}{\parbox{0.2\textwidth}{
\maindm{p=\frac{y}{1+y}}
}}}

\visible<7-8>{\placebox{0.843}{0.24}{\parbox{0.3\textwidth}{
\maindm{\hspace*{1em}=\frac{p_0e^{r_yt}}{1+p_0\left(e^{r_yt}-1\right)}}
}}}

\visible<10->{\placebox{0.81}{0.60}{\parbox{0.2\textwidth}{
\maindm{R_0\sup{mut}/R_0\sup{wild} = \tau r_y+1}
}}}

\visible<11->{\placebox{0.83}{0.50}{\parbox{0.3\textwidth}{
\maindm{R_0=(1-p) R_0\sup{wild} + pR_0\sup{mut}}
}}}

\visible<12->{\placebox{0.83}{0.28}{\parbox{0.3\textwidth}{
\maindm{\Rightarrow \ R_0\sup{wild}, R_0\sup{mut} \ \text{bekannt}}
}}}

\visible<12->{\placebox{0.836}{0.20}{\parbox{0.3\textwidth}{
\maindm{\Rightarrow \ R_0(t) \ \text{prognostizieren}}
}}}




}

\end{document}
%##############################################################


%##############################################################
%\frame{\frametitle{}
\frame{ 
%##############################################################


\placebox{0.50}{0.38}
 {\figSimple{1.10\textwidth}{figsCorona/screen2_2021_02_21_crop.png}}

% make images pale (see demo_makePale.tex)
%\makePale{opacity}{centerXrel}{centerYrel}{wrel}{hrel}
% e.g., \makePale{0.8}{0.5}{0.5}{1}{0.3}

\makePale{0.7}{0.48}{0.57}{1.2}{1.1}

\placebox{0.50}{0.90}
  {\myheading{Covid-19 Mutation Dynamics}}
\placebox{0.50}{0.82}{\myheading{mit \texttt{traffic-simulation.de}}}
\placebox{0.50}{0.74}{\mysubheading{Martin Treiber, TU Dresden}}

\placebox{0.50}{0.38}
 {\parbox{0.8\textwidth}{
  \bi
    \pause \item 1. Simulator
    \pause \item 2. Model
    \pause \item 3. 
    \pause \item 4. 
  \ei
}}

}

%##############################################################
\frame{\frametitle{Macroscopic Dynamics of new Virus Strains}
%##############################################################

\bi
\item \bfdef{Reproduction period $\tau$}: average time interval between
 two infection generations\\[1ex]
\pause \item \bfdef{Actively infected \#persons $x$} in the 'E' or 'I'
  states\\[1ex]
\pause \item \bfdef{Base reproduction numbers $R_0\sup{wild}$ 
and $R_0\sup{mut}$ }of
  the wild type and the mutated strain (e.g., B.1.1.7),
respectively\\[1ex]
\pause \item \bfdef{Penetration rate $p$} of the mutated strain
\ei

\vspace{2em}

\pause Dynamics without immunity and measures:

\maintextbox{0.5\textwidth}{
\vspace{-0.5em}
\begin{align*}
x(t+\tau) &=
  R_0 x \stackrel{!}{=} (1-p) R_0\sup{wild} x+pR_0\sup{mut}x\\
& \Rightarrow 
R_0=(1-p) R_0\sup{wild} + pR_0\sup{mut}
\end{align*}
\vspace{-1em}
}

}

%##############################################################
\frame{\frametitle{Macroscopic Dynamics of new Virus Strains II}
%##############################################################

\vspace{1em}

Dynamics of the \bfdef{percentage $p$} of the new strain:
\bdm
p(t+\tau)=\frac{x\sup{mut}(t+\tau)}{x\sup{wild}(t+\tau)}
= \frac{pR_0\sup{mut}}{(1-p)R_0\sup{wild}}
\edm

\pause Dynamics of the \bfdef{odds ratio} $y=p/(1-p)$:
%\bdm
%\left(\frac{p}{1-p}\right)_{t+\tau}=\frac{R_0\sup{mut}}{R_0\sup{wild}}
%\left(\frac{p}{1-p}\right)_{t}
%\edm
%Or, with $y=p/(1-p)$:
\bdm
y(t+\tau)=\frac{R_0\sup{mut}}{R_0\sup{wild}} \ y(t)
\edm

\pause Formulation as a \bfdef{differential equation}:
\be
\label{dydt}
\abl{y}{t} \approx \frac{y(t+\tau)-y(t)}{\tau}
=\frac{1}{\tau}\left(\frac{R_0\sup{mut}}{R_0\sup{wild}}-1\right)y 
\stackrel{!}{=} r_y y
\ee

\pause resulting \bfdef{replacement dynamics}:
\bdm
p=\frac{y}{1+y}, \quad
y(t)=y_0e^{r_yt}, \quad
\maindmIntext{p(t)=\frac{y_0e^{r_yt}}{1+y_0e^{r_yt}}
=\frac{p_0e^{r_yt}}{1+p_0\left(e^{r_yt}-1\right)}}
\edm
}

%##############################################################
\frame{\frametitle{Parameters from Data}
%##############################################################

\mysubsubheading{Given:}
{\small
\bi
\item Observed strain fractions $p_1$ and $p_2$ at the times  $t_1$
  and $t_2$
\item Observed or simulated base reproduction number $R_0$ at time $t$
\ei
\vspace{1em}

\pause
\mysubsubheading{Wanted:} 
dynamic parameters $r_y$, $R_0\sup{wild}$ and
  $R_0\sup{mut}$, and all the rest

\vspace{1em}

\pause
\mysubsubheading{Solution:} 

\bi
\item use the relation $y(t)=y_0e^{r_y(t-t_0)}$ to estimate
the growth rate  $r_y$:
\bdm
r_y \approx (\ln y(t_2)-\ln y(t_1))/(t_2-t_1)
\edm
\pause \item Use Relation~\eqref{dydt} to estimate the reproduction
number ratio: 
\bdm
R_0\sup{mut}/R_0\sup{wild} \approx \tau r_y+1
\edm

\pause \item Use the observed/simulated total base reproduction number to
  estimate the $R_0$'s individually: 
\maindm{
R_0\sup{wild}=\frac{R_0(t)}{1+p(t)\tau r_y}, \quad
R_0\sup{mut}=(\tau r_y+1)R_0\sup{wild}
}
\ei
}
}
 

%##############################################################
\frame{\frametitle{2. Modellierung}
%##############################################################

\placebox{0.5}{0.80}{\parbox{0.9\textwidth}{Sowohl die Mikro- als auch
    die Makrosimulation betrachten verschiedene 
Infektions\emph{phasen} und deren \emph{\"Uberg\"ange}
}}



\visible<2>{
 \placebox{0.5}{0.29}{\figSimple{1.08\textwidth}{figsCorona/SEIRsim02.png}}}

\visible<3>{
 \placebox{0.5}{0.29}{\figSimple{1.08\textwidth}{figsCorona/SEIRsim03.png}}}

\visible<4>{
 \placebox{0.5}{0.29}{\figSimple{1.08\textwidth}{figsCorona/SEIRsim04.png}}}

\visible<5>{
 \placebox{0.5}{0.29}{\figSimple{1.08\textwidth}{figsCorona/SEIRsim06.png}}}


% muss drunter sein, da das Bild den ganzen bereich ausfuellt!
\placebox{0.5}{0.60}{\parbox{0.9\textwidth}{
\benum
\visible<2->{\item Infizierbar (\emph{Susceptible, S})}
\visible<3->{\item Infiziert, noch nicht ansteckend (\emph{Exposed, E})}
\visible<4->{\item Infiziert, ansteckend (\emph{Infected, I})}
\visible<5->{\item Nach der ansteckenden Phase (\emph{Recovered/Removed, R})}
\eenum
}}

}



%##############################################################
\frame{\frametitle{2.2 Corona-Simulation.de: Modifiziertes SEIR-Makromodell} 
%##############################################################


\visible<1-2>{\placebox{0.5}{0.62}{
  \figSimple{1.05\textwidth}{figsCorona/SEIRsim06.png}}}

\visible<3->{\placebox{0.5}{0.62}{
  \figSimple{1.05\textwidth}{figsCorona/SEIRsim06a.png}}}

\placebox{0.53}{0.83}{\parbox{1.0\textwidth}{
  Die dynamischen Gr\"o\3en sind \emph{Anteilswerte} der Gesamtbev\"olkerung}}


\placebox{0.5}{0.20}{\parbox{1.0\textwidth}{
{\small
\bi
\visible<2->{\item Jeder Ansteckende infiziert nach 2-10 Tagen $R_0$
  andere Personen \emph{falls alle anderen noch ansteckbar (S) sind}}
\visible<3->{\item Bereits infizierte,
  ansteckende, geheilte \emph{oder geimpfte} Personen sind nicht
  ansteckbar (Annahme!), so dass der \emph{effektive}
    Reproduktionsfaktor $R$ u.U. deutlich geringer ist}
\visible<4->{\item Sobald ein Infizierter nicht mehr ansteckend ist,
  ist er aus der Infektionsdynamik raus, die weiteren Phasen sind
  nicht relevant}
\ei
}
}}

}


%##############################################################
\frame{\frametitle{3. Infektionen \emph{vs.} Beobachtungen} 
%##############################################################


\visible<1>{
 \placebox{0.5}{0.60}{\figSimple{1.08\textwidth}{figsCorona/SEIRsim06.png}}}

\visible<2>{
 \placebox{0.5}{0.60}{\figSimple{1.08\textwidth}{figsCorona/SEIRsim07.png}}}

\visible<3>{
 \placebox{0.5}{0.60}{\figSimple{1.08\textwidth}{figsCorona/SEIRsim08.png}}}

\visible<4>{
 \placebox{0.5}{0.60}{\figSimple{1.08\textwidth}{figsCorona/SEIRsim09.png}}}

\visible<5>{
 \placebox{0.5}{0.60}{\figSimple{1.08\textwidth}{figsCorona/SEIRsim10.png}}}

\visible<6>{
 \placebox{0.5}{0.60}{\figSimple{1.08\textwidth}{figsCorona/SEIRsim11.png}}}

\visible<7->{
 \placebox{0.5}{0.60}{\figSimple{1.08\textwidth}{figsCorona/SEIRsim12.png}}}



\placebox{0.5}{0.20}{\parbox{0.9\textwidth}{
{\small
\bi

\item \visible<1->{Das Infektionsgeschehen: Was ist beobachtbar?}
\visible<2->{\item Prinzipiell E- und I-Zustand mit PCR oder
  Antigentests, R mit Anik\"orpertest (und nat\"urlich Todesf\"alle)}
\visible<3->{\item Test-Sensitivit\"at $1-\alpha$=Prob\,(positiv $|$ infiziert)}
\visible<4->{\item Test-Spezifizit\"at $1-\beta$
            =Prob\,(negativ $|$ nicht infiziert); zwischen
            \unit[80]{\%} und \unit[99]{\%} der Getesteten sind
            \emph{nicht} infiziert!}
\ei
}
}}

}



%##############################################################
\frame{\frametitle{4. Anwendung: Szenarienprojektion} 
%##############################################################

\figSimple{0.95\textwidth}{figsCorona/screen2_2021_02_21.png}

}



\end{document}

