


%#############################################
% New pdflatex format,
% only png, jpg (or svg?) images but positioning now ok
% template in ~/tex/inputs/template_folien.tex
% [TODO now specialized to Vkoem_Ma; make general]
%#############################################


\documentclass[mathserif,aspectratio=1610]{beamer}
%\documentclass[mathserif,handout,aspectratio=1610]{beamer}
%\usepackage{beamerthemeshadow}
\input{$HOME/tex/inputs/defsSkript}  %$
\input{$HOME/tex/inputs/styleBeamerCorona}   %$

\usepackage{graphicx}


%##############################################################

\begin{document}

\section*{Impfdurchbruchsrate vs. Effizienz der Impfung}

%##############################################################
\frame{\frametitle{}
%##############################################################



%\placebox{0.50}{0.42}
% {\figSimple{1.10\textwidth}{figsCorona/coronaSimulationDe_1.png}}

% make images pale (see demo_makePale.tex)
%\makePale{opacity}{centerXrel}{centerYrel}{wrel}{hrel}
% e.g., \makePale{0.8}{0.5}{0.5}{1}{0.3}

%\makePale{0.7}{0.48}{0.57}{1.2}{1.1}

\placebox{0.50}{0.44}{\parbox{1.0\textwidth}{
\hspace{0.2\textwidth} \mysubheading{Plakatives Beispiel}
{\small
\bi
\item {\small 1\,000 Leute, darunter $p=\unit[80]{\%}$, also 800, vollst\"andig geimpft}
\figSimple{0.06\textwidth}{figsCorona/syringe.jpg}
\hspace{-0.05\textwidth}
\figSimple{0.06\textwidth}{figsCorona/syringe.jpg}
\\[-1em]

\pause \item $1-p=\unit[20]{\%}$ bzw 200 sind gar
nicht/unvollst\"andig geimpft 
\hspace{0.02\textwidth} 
{\Huge{$\mathbf{\red{\times}}$}} 
\hspace{-0.04\textwidth}
\visible<2->{\figSimple{0.06\textwidth}{figsCorona/syringe.jpg}}

\pause \item Effizienz der Impfung 
$E=1-\frac{P(\text{am n\"achsten Tag erkrankt}|\text{geimpft})}
     {P(\text{am n\"achsten Tag erkrankt}|\text{ungeimpft})}
     = \unit[75]{\%}$

\pause \item Die Erkrankungswahrscheinlichkeit Ungeimpfter f\"ur den
n\"achsten Tag sei \unit[1]{\%}, also im Mittel 2 Leute

\pause \item Bei einer Effizienz von \unit[75]{\%} ist damit die
Erkrankungswahrscheinlichkeit Geimpfter \unit[0.25]{\%}, also bei
800 Leuten ebenfalls 2

\pause \item Damit ist die Impfdurchbruchsquote
\bdm
p\subb{durch}=\frac{\text{\# F\"alle bei vollst\"andig Geimpften}}
{\text{Gesamtzahl der F\"alle}}=\frac{2}{2+2}=\uu{\unit[50]{\%}}
\edm

\pause \item Allg. Formel im Simulator:
\vspace{-1em}
\bdm
p\subb{durch}=\frac{p(1-E)}{1-pE}
\edm

\ei


}}}


}


\end{document}













