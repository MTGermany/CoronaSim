


%#############################################
% New pdflatex format,
% only png, jpg (or svg?) images but positioning now ok
% template in ~/tex/inputs/template_folien.tex
% [TODO now specialized to Vkoem_Ma; make general]
%#############################################


\documentclass[mathserif,aspectratio=1610]{beamer}
%\documentclass[mathserif,handout,aspectratio=1610]{beamer}
%\usepackage{beamerthemeshadow}
\input{$HOME/tex/inputs/defsSkript}  %$
\input{$HOME/tex/inputs/styleBeamerCorona}   %$

\usepackage{graphicx}


%##############################################################

\begin{document}

\section*{Omicron Dynamics}

%##############################################################
\frame{\frametitle{Latest observation 1}
%##############################################################

\placebox{0.35}{0.24}
 {\figSimple{0.60\textwidth}{figsCorona/2021_12_15_UK.png}}
\placebox{0.30}{0.65}
 {\figSimple{0.50\textwidth}{figsCorona/2021_12_15_DK_abs.png}}

\placebox{0.80}{0.47}{\parbox{0.40\textwidth}{
\bi
\item Data from Danmark and London
\item The Delta and Omicron variants coexist without
  directly affecting each other 
\item Indirect interaction via competing for common ressources, i,e.,
  \emph{first come, first served}
\ei
}}

}

%##############################################################
\frame{\frametitle{Latest observation 2}
%##############################################################

\placebox{0.28}{0.46}
 {\figSimple{0.60\textwidth}{figsCorona/2021_12_15_DK_rel.png}}

\placebox{0.76}{0.68}{\parbox{0.46\textwidth}{
\bi
\item The share of Omicron can be well described by a logistic
  function with growth rate $r$
\ei
}}

}

%##############################################################
\frame{\frametitle{Latest observation 2: Update}
%##############################################################

\visible<1-2>{\placebox{0.28}{0.46}
 {\figSimple{0.60\textwidth}{figsCorona/2021_12_21_DK_rel.png}}}
\visible<3>{\placebox{0.28}{0.46}
 {\figSimple{0.60\textwidth}{figsCorona/2021_12_21_DK_odds.png}}}
\visible<4>{\placebox{0.28}{0.46}
 {\figSimple{0.60\textwidth}{figsCorona/2021_12_21_DK_logodds.png}}}
\visible<5->{\placebox{0.28}{0.46}
 {\figSimple{0.60\textwidth}{figsCorona/2021_12_21_DK_rel.png}}}

\placebox{0.76}{0.42}{\parbox{0.46\textwidth}{
{\small
\bi
\visible<1->{\item Data update.}
\visible<2->{How to get the curve?}
\visible<3->{\item Transform the observed Omicron share $p$ into the
  \bfdef{odds ratio} $y=p/(1-p)$}
\visible<3->{\item From Observation~1 (coexistence), it follows that the odds ratio
  grows exponentially:
\vspace{-0.5em}
\bdm
y(t)=y_0 \, e^{rt}
\edm
\vspace{-1.5em}}
\visible<4->{\item This means, the \bfdef{log-odds} are essentially
  linear in time:
\vspace{0em}
\bdm
\ln y(t)=\ln y_0+rt
\edm
\vspace{-1.5em}}

\visible<5->{\item Transforming back gives the s-shaped predicted
  Omicron share (logistic function)
\vspace{-1.5em}
\bdm
p(t)=\frac{y(t)}{1+y(t)}
\edm
}
\ei
}

}}

}


%##############################################################
\frame{\frametitle{Relation between the logistic growth rate $r$ and
    the reproduction numbers}
%##############################################################

Assumptions:
\bi
\item Neither positive nor negative \bfdef{cross effects}: Each variant acts
  on its own (using common ressources of susceptible humans)

\pause\item The Delta and Omicron variants have different \bfdef{base
reproduction numbers} $R_{10}$ and $R_{20}$ and different \bfdef{generation
times} $T_1$ and $T_2$, respectively (e.g., $R_{10}=5$, $T_1=\unit[5]{days}$,
$T_2=\unit[4]{days}$) 

\pause\item The \bfdef{immunities} $I_1$ and $I_2$ (including vaccinations and
past infections) against Delta and Omicron are generally different

\pause\item The \bfdef{reduction factors $f\subb{m}$} by isolation measures and
the \bfdef{seasonal factor $f\subb{s}$} are common

\pause\item All factors influencing the effective reproduction number
$R$ are multiplicative
\ei

\pause
\maineq{baseDyn}{
\Rightarrow 
\begin{array}{ll}
x_1(t_0+T_1)=R_1x_1(t_0)=R_{10}(1-I_1)f\subb{m}f\subb{s} x_1(t_0),\\
x_2(t_0+T_2)=R_2x_2(t_0)=R_{20}(1-I_2)f\subb{m}f\subb{s} x_2(t_0)
\end{array}
}

}

%##############################################################
\frame{\frametitle{Relation between the logistic growth rate $r$ and
    $R_1$ and $R_2$ (ctned)}
%##############################################################
\vspace{1em}

Assuming continuous infections (slowly varying rates), we can write
~\eqref{baseDyn} as
\bdm
x_1(t)=x_1(0) R_1^{\ t/T_1}=x_1(0)\exp\left(\frac{t}{T_1}\ln R_1\right)
\equiv x_1(0)\exp(r_1t), 
\edm
\pause
$x_2(t)$ likewise, or
\be
\label{dotx1x2}
\dot{x}_1=r_1t, \ r_1=\frac{\ln R_1}{T_1}, \quad
\dot{x}_2=r_2t, \ r_2=\frac{\ln R_2}{T_2}
\ee

\pause
How does the odds $y=x_2/x_1$ evolve?
\vspace{-1em}

\bdma
\dot{y} &=& \abl{}{t}\left(\frac{x_2}{x_1}\right) 
\pause
 =\frac{\dot{x_2}}{x_1}-\frac{x_2}{x_1^2}\dot{x_1}\\
\pause &=& \frac{r_2x_2}{x_1}-\frac{x_2}{x_1^2}r_1x_1=(r_2-r_1)y
\edma

\pause
\maineq{r}{\Rightarrow \quad 
 r=r_2-r_1=\frac{\ln R_2}{T_2}-\frac{\ln R_1}{T_1}}

\pause
\footnotesize{Special case $T_2=T_1=T$: \quad
$R_2=R_1\exp(rT)$}

}


%##############################################################
\frame{\frametitle{Determining the Omicron base reproduction rate from
  the logistic growth rate $r$}
%##############################################################

Just use Relation~\eqref{r} and insert the definitions of $R_1$ and
$R_2$ from~\eqref{baseDyn}

\pause
After some manipulations ...
\maineq{R20}{
R_{20}=\exp(rT_2) \, f\subb{m}^{\gamma-1} \, f\subb{s}^{\gamma-1} \,
\frac{\left(R_{10}(1-I_1)\right)^\gamma}{1-I_2}, \quad
\gamma=\frac{T_2}{T_1}
}

\pause
\bi
\item For equal generation times $T_1=T_2=T$, the measures and the
  seasonal effects drop out and $r$ depends only on the past infection
  and vacination immunities (remains time dependent since the
  immunities change):
\bdm
T_1=T_2=T \quad \Rightarrow R_{20}=e^{rT}\ \frac{R_{10}(1-I_1)}{1-I_2}
\edm
{\scriptsize{Example: $e^{rT}=3$, $R_{10}=4$, $I_1=0.6$, $I_2=0.2$, $R_{20}=3/2 R_{10}=6$}}

\pause \item With neither immunities nor measures nor season effects but
  $T_2=0.5T_1$ ($\gamma=0.5$), we have  
$R_{20}=e^{rT_2}\sqrt{R_{10}}$, e.g., for $e^{rT_2}=2$ and
  $R_{01}=4$, we have $R_{02}=R_{01}=4$

\pause \item With measures/season effects and immunities as above, we
may have
$R_{02}<R_{01}$
\ei

}


%##############################################################
\frame{\frametitle{Effective infection growth rate and reproduction number}
%##############################################################

The \bfdef{effective growth rate $r\subb{eff}$} of the infection dynamics (not to
be confused with the logistic growth rate $r$ of the Omicron shares $p$)
comes directly
from~\eqref{dotx1x2}:

\bdm
\dot{x}=\dot{x}_1+\dot{x}_2=r_1x_1+r_2x_2=
\big[ (1-p)r_1+pr_2\big] x \equiv r\subb{eff}x
\edm

\pause
\bfdef{Effective
reproduction number $R\subb{eff}$} by association
$r\subb{eff}\equiv\ln R\subb{eff}/T_1$:

\maineq{Reff}{
 \ln R\subb{eff}=(1-p)\ln R_1
 + \frac{p}{\gamma} \ln R_2}

\pause
\bi
\item Because $1/\gamma=T_1/T_2>1$, influence factors, e.g., measures,
  have a more sensitive effect on Omicron 
  than om Delta: If $T_1/T_2=1/\gamma=2$ and measures lead to a
  factor $1/\sqrt{2}\approx 0.7$ on Delta ($R_1$), they simultaneously
  lead to a factor 1/2 on Omicron ($R_2$)
\vspace{0.5em}

\pause\item If, at a certain time, the true Omicron share $p$, the
effective reproduction number $R\subb{eff}$, and the logistic growth
rate $r$ are known (all three can be estimated), and the generation
time ratio
$\gamma=T_2/T_1$ as well as the total immunities $I_1$ and $I_2$ and
the effects of the measures and the season at 
this time can be estimated, the Eqs~\eqref{baseDyn},
\eqref{R20}, and~\eqref{Reff} allow for a simultaneous estimation of
$R_{10}$ and $R_{20}$
\ei
}



%##############################################################
\frame{\frametitle{Assumed efficiency against Delta infections}
%##############################################################

\placebox{0.40}{0.42}
 {\figSimple{0.75\textwidth}{figsCorona/2021_12_vaccBoostEfficiencyDelta.png}}
\placebox{0.85}{0.42}{\parbox{0.20\textwidth}{
  ``First vaccinated-\\first boostered''\\principle}}
\placebox{0.50}{0.85}{\myHyperlink{https://assets.publishing.service.gov.uk/government/uploads/system/uploads/attachment_data/file/1043807/technical-briefing-33.pdf}{UK
  Technical briefing 33}}

}

%##############################################################
\frame{\frametitle{Assumed efficiency against Omicron infections}
%##############################################################
\placebox{0.40}{0.44}
 {\figSimple{0.75\textwidth}
 {figsCorona/2021_12_vaccBoostEfficiencyOmicron.png}}
\placebox{0.85}{0.44}{\parbox{0.20\textwidth}{
  Only fresh\\full vaccinations\\or boosters\\help against\\Omicron}}
\placebox{0.50}{0.85}{\myHyperlink{https://assets.publishing.service.gov.uk/government/uploads/system/uploads/attachment_data/file/1043807/technical-briefing-33.pdf}{UK
  Technical briefing 33}}

}

%##############################################################
\frame{\frametitle{Assumed immunity by infections}
%##############################################################
\placebox{0.50}{0.55}{\parbox{0.70\textwidth}{
\bi
\item \unit[100]{\%} immunity of Delta against Delta reinfections\\[1em]
\pause\item \unit[100]{\%} 
immunity of Omicron against Omicron reinfections\\[1em]
\pause\item \unit[100]{\%} no cross immunity (people can get both Delta and
  Omicron infections)
\ei}}

}


%##############################################################
\frame{\frametitle{Determining the population immunities I: vaccinations}
%##############################################################
Here, I make following assumptions
\bi
\item Vaccination efficiency curves $I_1\supb{v}(\tau)$ and
$I_2\supb{v}(\tau)$ against Delta and Omega as shown,
\pause\item corresponding booster efficencies  $I_1\supb{b}(\tau)$ and
$I_2\supb{b}(\tau)$,
\pause\item \emph{First vaccinated-first boostered} 
\ei
\pause Since the protection depends on the vaccination times, I sum up the
different histories weighted with the past daily vaccination and
booster rates $r^v_{t'}$ and $r^b_{t'}$ (fraction of the population
per day): 
\bdm
I_1\supb{vacc}(t)=\sum_{t'=t_v}^t r^v_{t'}I_1\supb{v}(t-t')
+\sum_{t'=t_b}^t r^b_{t'}I_1\supb{b}(t-t')
\edm
where $t_b$ is the time of the first booster shot, and $t_v$ the oldest
time of the first
vaccination of any person who is not yet boostered.
\vspace{1em}

The vaccination immunity $I_2$ is calculated in analogy.
}

%##############################################################
\frame{\frametitle{Determining the population immunities II:
    infections and total}
%##############################################################
\bi
\item Everybody can only be infected once with any variant but there
  is no cross immunity, so the immunity is just equal to the total
  percentage $X_1$ and $X_2$ of people infected with either variant: 
\bdm
I_1^x=X_1, \quad I_2^x=X_2
\edm

\emph{Notice}: $X_i$ is not just the cumulated number of cases divided
by the population because any infection, whether detected or not
detected, counts

\vspace{1em}

\pause \item There is no correlation between vaccinations and infections:
\maineq{I1I2}{
1-I_1=(1-I_1^v)(1-I_1^x), \quad
1-I_2=(1-I_2^v)(1-I_2^x)
}
\ei

}

%##############################################################
\frame{\frametitle{Simulation}
%##############################################################

\placebox{0.50}{0.52}
 {\figSimple{0.95\textwidth}{figsCorona/2021_12_Omicron_screenshot1.png}}

\placebox{0.50}{0.11}{\parbox{0.90\textwidth}{
  All items $I_1$, $I_2$, $p$, $R_{10}$, $R_{20}$, $f\subb{season}$
  and $f\subb{stringency}$ are displayed in the simulation}}

}



\end{document}


%##############################################################
%##############################################################







% make images pale (see demo_makePale.tex)
%\makePale{opacity}{centerXrel}{centerYrel}{wrel}{hrel}
% e.g., \makePale{0.8}{0.5}{0.5}{1}{0.3}

%\makePale{0.7}{0.48}{0.57}{1.2}{1.1}

%\placebox{0.50}{0.44}{\parbox{1.0\textwidth}{
%\hspace{0.2\textwidth} \mysubheading{Plakatives Beispiel}



%##############################################################
\frame{\frametitle{Putting it all together}
%##############################################################
\bi
\item[(1)] The spreading of the fraction $p$ new variant
  (Omicron) at the cost of the old variants (Delta) is logistic:
\maindm{
p(t)=\frac{y_0 e^{rt}}{1+y_0 e^{rt}}, \quad y_0=\frac{p_0}{1-p_0}
}
\vspace{1em}


\pause \item[(2)] The growth rate $r$ of the logistic growth depends on
\vspace{1ex}

\bi
\item the base reproduction numbers $R_{10}$ and $R_{20}$,
%\ei\ei}\end{document}
\pause \item the total population immunities (vaccination, boosters,
  infections) $I_1$ and $I_2$ against each variant,
\pause \item the generation time $T$ of the infections (assumed to be
  equal, 5~days):
\ei\vspace{1ex}

\pause according to
\maineq{r}{
r=\frac{1}{T}\left[\frac{R_{20}(1-I_2)}{R_{10}(1-I_1)}-1\right]
}
\vspace{1ex}

\emph{Notice:} Since the $I_i$ are time dependent, so is the growth rate $r$
\ei

}

%##############################################################
\frame{\frametitle{Putting it all together (ctned)}
%##############################################################
\bi
\item[(3)] Assuming no cross effects, 
the effective $R_t$ value caused by the mixture of the Delta and
  Omicron viruses depends on
\bi
\item the Omicron percentage $p$
\item the base reproduction numbers $R_{10}$ and $R_{20}$ of Delta
  and Omicron, respectively,
\item the immunity escape fractions $(1-I_1)$ and $(1-I_2)$,
\item the seasonal multiplicator $f\supb{season}$ %(which I have calibrated)
\item the stringency (lockdown) multiplicator $f\supb{stringency}$:
\ei

\maineq{Rt}{R_t=\left[(1-p)R_{10}(1-I_1)+pR_{20}(1-I_2)\right]
 f\subb{season} f\subb{stringency}}
\vspace{0em}

\bi
\pause \item
\emph{Notice:} Both the \red{growth rate $r$} of the increase of the Delta
share and the \red{growth rate $(R_t-1)/T$} of the actual infection wave
depend on $P_1= R_{10}(1-I_1)$ and $P_2=R_{20}(1-I_2)$ such that a positive
(negative) value of $r$ implies an increase (decrease) of $R_t$\\[1ex]
\pause\item A positive Omicron spreading $r$ does \red{not} imply that the new
  variant is more 
  infectious; only the products $P_1$ and $P_2$ matter\\[1ex]
\pause\item A \red{positive} spreading $r$ can be related to a 
\red{negative} infection
  growth $(R_t-1)/T$ both before \emph{and} after the new variant
  dominates since, for given $I_1$ and $I_2$, $r$ is only related to
  the \emph{ratio} $R_{20}/R_{10}$ of the base immunities
  $\Rightarrow$ next slide
\ei
\ei

}


%##############################################################
\frame{\frametitle{Estimating the base reproduction numbers}
%##############################################################

Assume at time $t=t^*$ given population immunities $I_1^*$ and $I_2^*$
(see later), and estimations of the Omega share $p^*$, Omicron
spreading rate $r^*$, and effective reproduction number $R^*_t$ of the
mixture. Then, we can use Eq~\eqref{r} to obtain the ratio

\maineq{R20R10}
  {\frac{R_{20}}{R_{10}}=\frac{(r^*T+1)(1-I_1^*)}{1-I_2^*}}

\pause and, with Eq~\eqref{Rt} determine the base reproduction numbers
individually:

\maineq{R10}
  {R_{10}=\frac{R^*_t}{(1+p^*r^*T)f\subb{season} f\subb{stringency}}}
\vspace{1em}

For $t>t^*$, I assume fixed base reproduction numbers and calculate
the future Omicron spreading and the future wave using
\eqref{r} and \eqref{Rt} with time varying $I_1$, $I_2$, $p$,
$f\subb{season}$, and $f\subb{stringency}$

}







