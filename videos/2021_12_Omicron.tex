


%#############################################
% New pdflatex format,
% only png, jpg (or svg?) images but positioning now ok
% template in ~/tex/inputs/template_folien.tex
% [TODO now specialized to Vkoem_Ma; make general]
%#############################################


\documentclass[mathserif,aspectratio=1610]{beamer}
%\documentclass[mathserif,handout,aspectratio=1610]{beamer}
%\usepackage{beamerthemeshadow}
\input{$HOME/tex/inputs/defsSkript}  %$
\input{$HOME/tex/inputs/styleBeamerCorona}   %$

\usepackage{graphicx}


%##############################################################

\begin{document}

\section*{Omicron Dynamics}

%##############################################################
\frame{\frametitle{Latest observation 1}
%##############################################################

\placebox{0.30}{0.65}
 {\figSimple{0.50\textwidth}{figsCorona/2021_12_20_DK_abs.png}}
\placebox{0.35}{0.22}
 {\figSimple{0.60\textwidth}{figsCorona/2021_12_15_UK.png}}

\placebox{0.80}{0.47}{\parbox{0.40\textwidth}{
\bi
\item The Delta and Omicron variants coexist without
  directly affecting each other 
\item Indirect interaction via competing for common ressources, i,e.,
  \emph{first come, first served}
\ei
}}

}

%##############################################################
\frame{\frametitle{Latest observation 2}
%##############################################################

\placebox{0.28}{0.46}
 {\figSimple{0.60\textwidth}{figsCorona/2021_12_20_DK_rel.png}}

\placebox{0.76}{0.42}{\parbox{0.46\textwidth}{
\bi
\item The share of Omicron can be well described by a logistic
  function with growth rate $r$
\pause \item First transform the observed Omicron share $p$ into the
\bfdef{odds ratio} $y=p/(1-p)$
\pause \item From Observation~1 (coexistence), it follows that the odds ratio
  grows exponentially:
\vspace{-0.5em}
\bdm
y(t)=y_0 \, e^{rt}
\edm
\vspace{-1.5em}
\pause \item Transforming back gives the s-shaped predicted Omicron share
  (logistic function)
\vspace{-1em}
\bdm
p(t)=\frac{y(t)}{1+y(t)}
\edm
\ei
}}

}


%##############################################################
\frame{\frametitle{Assumed efficiency against Delta infections}
%##############################################################
\placebox{0.40}{0.46}
 {\figSimple{0.75\textwidth}{figsCorona/2021_12_vaccBoostEfficiencyDelta.png}}
\placebox{0.85}{0.46}{\parbox{0.20\textwidth}{
  ``First vaccinated-\\first boostered''\\principle}}

}

%##############################################################
\frame{\frametitle{Assumed efficiency against Omicron infections}
%##############################################################
\placebox{0.40}{0.46}
 {\figSimple{0.75\textwidth}
 {figsCorona/2021_12_vaccBoostEfficiencyOmicron.png}}
\placebox{0.85}{0.46}{\parbox{0.20\textwidth}{
  Only fresh\\full vaccinations\\or boosters\\help against\\Omicron}}

}

%##############################################################
\frame{\frametitle{Assumed immunity by infections}
%##############################################################
\placebox{0.50}{0.55}{\parbox{0.70\textwidth}{
\bi
\item \unit[100]{\%} immunity of Delta against Delta reinfections\\[1em]
\pause\item \unit[100]{\%} 
immunity of Omicron against Omicron reinfections\\[1em]
\pause\item \unit[100]{\%} no cross immunity (people can get both Delta and
  Omicron infections)
\ei}}

}

%##############################################################
\frame{\frametitle{Putting it all together}
%##############################################################
\bi
\item[(1)] The spreading of the fraction $p$ new variant
  (Omicron) at the cost of the old variants (Delta) is logistic:
\maindm{
p(t)=\frac{y_0 e^{rt}}{1+y_0 e^{rt}}, \quad y_0=\frac{p_0}{1-p_0}
}
\vspace{1em}


\pause \item[(2)] The growth rate $r$ of the logistic growth depends on
\vspace{1ex}

\bi
\item the base reproduction numbers $R_{10}$ and $R_{20}$,
%\ei\ei}\end{document}
\pause \item the total population immunities (vaccination, boosters,
  infections) $I_1$ and $I_2$ against each variant,
\pause \item the generation time $T$ of the infections (assumed to be
  equal, 5~days):
\ei\vspace{1ex}

\pause according to
\maineq{r}{
r=\frac{1}{T}\left[\frac{R_{20}(1-I_2)}{R_{10}(1-I_1)}-1\right]
}
\vspace{1ex}

\emph{Notice:} Since the $I_i$ are time dependent, so is the growth rate $r$
\ei

}

%##############################################################
\frame{\frametitle{Putting it all together (ctned)}
%##############################################################
\bi
\item[(3)] Assuming no cross effects, 
the effective $R_t$ value caused by the mixture of the Delta and
  Omicron viruses depends on
\bi
\item the Omicron percentage $p$
\item the base reproduction numbers $R_{10}$ and $R_{20}$ of Delta
  and Omicron, respectively,
\item the immunity escape fractions $(1-I_1)$ and $(1-I_2)$,
\item the seasonal multiplicator $f\sup{season}$ %(which I have calibrated)
\item the stringency (lockdown) multiplicator $f\sup{stringency}$:
\ei

\maineq{Rt}{R_t=\left[(1-p)R_{10}(1-I_1)+pR_{20}(1-I_2)\right]
 f\sub{season} f\sub{stringency}}
\vspace{0em}

\bi
\pause \item
\emph{Notice:} Both the \red{growth rate $r$} of the increase of the Delta
share and the \red{growth rate $(R_t-1)/T$} of the actual infection wave
depend on $P_1= R_{10}(1-I_1)$ and $P_2=R_{20}(1-I_2)$ such that a positive
(negative) value of $r$ implies an increase (decrease) of $R_t$\\[1ex]
\pause\item A positive Omicron spreading $r$ does \red{not} imply that the new
  variant is more 
  infectious; only the products $P_1$ and $P_2$ matter\\[1ex]
\pause\item A \red{positive} spreading $r$ can be related to a 
\red{negative} infection
  growth $(R_t-1)/T$ both before \emph{and} after the new variant
  dominates since, for given $I_1$ and $I_2$, $r$ is only related to
  the \emph{ratio} $R_{20}/R_{10}$ of the base immunities
  $\Rightarrow$ next slide
\ei
\ei

}


%##############################################################
\frame{\frametitle{Estimating the base reproduction numbers}
%##############################################################

Assume at time $t=t^*$ given population immunities $I_1^*$ and $I_2^*$
(see later), and estimations of the Omega share $p^*$, Omicron
spreading rate $r^*$, and effective reproduction number $R^*_t$ of the
mixture. Then, we can use Eq~\eqref{r} to obtain the ratio

\maineq{R20R10}
  {\frac{R_{20}}{R_{10}}=\frac{(r^*T+1)(1-I_1^*)}{1-I_2^*}}

\pause and, with Eq~\eqref{Rt} determine the base reproduction numbers
individually:

\maineq{R10}
  {R_{10}=\frac{R^*_t}{(1+p^*r^*T)f\sub{season} f\sub{stringency}}}
\vspace{1em}

For $t>t^*$, I assume fixed base reproduction numbers and calculate
the future Omicron spreading and the future wave using
\eqref{r} and \eqref{Rt} with time varying $I_1$, $I_2$, $p$,
$f\sub{season}$, and $f\sub{stringency}$

}

%##############################################################
\frame{\frametitle{Determining the population immunities I: vaccinations}
%##############################################################
Here, I make following assumptions
\bi
\item Vaccination efficiency curves $I_1\sup{v}(\tau)$ and
$I_2\sup{v}(\tau)$ against Delta and Omega as shown,
\pause\item corresponding booster efficencies  $I_1\sup{b}(\tau)$ and
$I_2\sup{b}(\tau)$,
\pause\item \emph{First vaccinated-first boostered} 
\ei
\pause Since the protection depends on the vaccination times, I sum up the
different histories weighted with the past daily vaccination and
booster rates $r^v_{t'}$ and $r^b_{t'}$ (fraction of the population
per day): 
\bdm
I_1\sup{vacc}(t)=\sum_{t'=t_v}^t r^v_{t'}I_1\sup{v}(t-t')
+\sum_{t'=t_b}^t r^b_{t'}I_1\sup{b}(t-t')
\edm
where $t_b$ is the time of the first booster shot, and $t_v$ the oldest
time of the first
vaccination of any person who is not yet boostered.
\vspace{1em}

The vaccination immunity $I_2$ is calculated in analogy.
}

%##############################################################
\frame{\frametitle{Determining the population immunities II:
    infections and total}
%##############################################################
\bi
\item Everybody can only be infected once with any variant but there
  is no cross immunity, so the immunity is just equal to the total
  percentage $X_1$ and $X_2$ of people infected with either variant: 
\bdm
I_1^x=X_1, \quad I_2^x=X_2
\edm

\emph{Notice}: $X_i$ is not just the cumulated number of cases divided
by the population because any infection, whether detected or not
detected, counts

\vspace{1em}

\pause \item There is no correlation between vaccinations and infections:
\maineq{I1I2}{
1-I_1=(1-I_1^v)(1-I_1^x), \quad
1-I_2=(1-I_2^v)(1-I_2^x)
}
\ei

}

%##############################################################
\frame{\frametitle{Simulation}
%##############################################################

\placebox{0.50}{0.52}
 {\figSimple{0.95\textwidth}{figsCorona/2021_12_Omicron_screenshot1.png}}

\placebox{0.50}{0.11}{\parbox{0.90\textwidth}{
  All items $I_1$, $I_2$, $p$, $R_{10}$, $R_{20}$, $f\sub{season}$
  and $f\sub{stringency}$ are displayed in the simulation}}

}



\end{document}




% make images pale (see demo_makePale.tex)
%\makePale{opacity}{centerXrel}{centerYrel}{wrel}{hrel}
% e.g., \makePale{0.8}{0.5}{0.5}{1}{0.3}

%\makePale{0.7}{0.48}{0.57}{1.2}{1.1}

%\placebox{0.50}{0.44}{\parbox{1.0\textwidth}{
%\hspace{0.2\textwidth} \mysubheading{Plakatives Beispiel}










