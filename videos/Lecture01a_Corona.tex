


%#############################################
% New pdflatex format,
% only png, jpg (or svg?) images but positioning now ok
% template in ~/tex/inputs/template_folien.tex
% [TODO now specialized to Vkoem_Ma; make general]
%#############################################


%\documentclass[mathserif]{beamer}
\documentclass[mathserif,handout]{beamer}
%\usepackage{beamerthemeshadow}
\input{$HOME/tex/inputs/defsSkript}  %$
\input{$HOME/tex/inputs/styleBeamerVkoekMa}   %$

\usepackage{graphicx}


%##############################################################

\begin{document}

\section*{1a: Covid-19 Dynamics}

%##############################################################
%\frame{\frametitle{}
\frame{ 
%##############################################################


\placebox{0.50}{0.42}
 {\figSimple{1.10\textwidth}{figsCorona/coronaSimulationDe_1.png}}

% make images pale (see demo_makePale.tex)
%\makePale{opacity}{centerXrel}{centerYrel}{wrel}{hrel}
% e.g., \makePale{0.8}{0.5}{0.5}{1}{0.3}

\makePale{0.7}{0.48}{0.57}{1.2}{1.1}

\placebox{0.50}{0.90}{\myheading{Traffic Econometrics Master's Course}}
\placebox{0.50}{0.82}{\mysubheading{Lecture 01a: Covid-19 Dynamics}}
\placebox{0.86}{0.05}{Martin Treiber} 

\placebox{0.50}{0.38}
 {\parbox{0.8\textwidth}{
  \bi
    \pause \item 1a.1 Simple Macroscopic Models (SI, SIR, SEIR, SIRM)
    \pause \item 1a.2 Microscopic Models
    \pause \item 1a.3 Down to Earth: Data-Related Issues
    \pause \item 1a.4 Simulation \texttt{Corona-simulation.de}
  \ei
}}

}




 \subsection*{1a.1 Simple macroscopic models}

%##############################################################
\frame{\frametitle{1a.1 Simple macroscopic models I: SI model}
%##############################################################

\bfdef{Compartemental models}: consider different status such as susceptible,
infected, or recovered and transitions between them

\bi
\item As in any macroscopic model on infection dynamics, the basic
  dynamic quantities are \emph{percentages of the population} (e.g., of a
  country) rather than individual persons

\pause \item Scale separation: The \emph{infection dynamics} is much faster than the
  rest of the \emph{population dynamics}
  (births, ``normal'' deaths, in- and outwards directed
  migration/moves) $\Rightarrow$ population number $N=\text{const.}$

\pause \item Two compartiments: any person can be either \emph{susceptible} to infection (S), or
  already infected (I) which includes actually ill, recovered, or
  dead. Particularly, there is no reverse transition I$\to$S
\visible<3->{\fig{0.6\textwidth}{figsCorona/SI.png}}
\ei

}

%##############################################################
\frame{\frametitle{SI model II}
%##############################################################


{\small
\bi
\item All infected persons become \emph{contagious instantaneously} and
  remain so all the time (notice the
  inconsistency to the point above)

\pause \item The \emph{rate of contagion} $\beta$ (\# persons per time unit if
  everybody else is~S) remains constant 
\ei
}

\pause
\maintextbox{0.6\textwidth}{{\large
$\Rightarrow \quad \begin{array}{ll}
\abl{S}{t} &=-\beta I S,\\[0.5em]
\abl{I}{t} &=+\beta I S
\end{array}$
}
\qquad
SI model
}
\bi
\item $S=N_S/N$: fraction of susceptible
\pause \item $I=N_I/N$: fraction of infected
\pause \item $\abl{}{t}(S+I)=0$ $\Leftrightarrow$ conservation of
  population number $N=\text{const.}$
\ei
\pause Rewrite with $S+I=1$:
\bdm
\abl{I}{t}=\beta I (1-I) 
\edm
$\Rightarrow$ \emph{classical model for limited growth} with saturation 1
}

%##############################################################
\frame{\frametitle{SI model III: Simulation}
%##############################################################

\placebox{0.5}{0.43}{
 \figSimple{0.95\textwidth}{figsCorona/modelSI.png}}

}



%##############################################################
\frame{\frametitle{SIR model}
%##############################################################

{\small
\bi
\item Unlike the situation in the SI model, infected people recover/die after an
  average  time
$1/\gamma$ thereby becoming \emph{no longer contagious} 
\pause \item Chained models for the transitions susceptible-infected (SI) and
  infected-recovered persons(IR), $R=$ fraction of recovered:
\ei

\vspace{-1em}
\visible<2->{\fig{0.8\textwidth}{figsCorona/SIR.png}}
\vspace{-1em}

%Chained model equations for $S$, $I$ and $R$ (fraction $N_R/N$ of
%recovered/removed persons): 

\pause
\maintextbox{0.6\textwidth}{{\large
$\begin{array}{ll}
\abl{S}{t} &=-\beta I S,\\[0.5em]
\abl{I}{t} &=+\beta I S - \gamma I,\\[0.5em]
\abl{R}{t} &=+\gamma I
\end{array}$
}
\qquad
SIR model
}
\vspace{-1ex}

\bi
\pause \item Conservation of the population number: $S+I+R=1$
\pause \itemAsk \colAsk{Show that the initial reproduction number is given by
  $R_0=\beta/\gamma$}\\[-1ex]
\pause \itemAnswer \colAnswer{\tiny{Initially ($S=1$), any infected person infects
  $\beta$ other persons per day but recovers after an\\[-1ex]exponentially
  distributed time $\tau_R \sim \text{Exp}(\gamma)$, so the
  average \#infected people $= \beta E(\tau_R)=\beta/\gamma$}}
\ei
}


}


%##############################################################
\frame{\frametitle{SI and SIR models: simulation}
%##############################################################

\placebox{0.5}{0.43}{
 \figSimple{0.95\textwidth}{figsCorona/modelSIR.png}}
\visible<2>{\placebox{0.5}{0.43}{
 \figSimple{0.95\textwidth}{figsCorona/modelSI.png}}}
\visible<3>{\placebox{0.5}{0.43}{
 \figSimple{0.95\textwidth}{figsCorona/modelSIR.png}}}
}


%##############################################################
\frame{\frametitle{SEIR model}
%##############################################################

{\small
\bi
\item Adds to the SIR model a finite \emph{incubation time} $\tau_I
  \sim \text{Exp}(\alpha)$ where people are infected but not yet
  contagious (``exposed'', E)
\pause \item Triple chain with $S+E+I+R=1$:
\ei

\vspace{-1em}
\visible<2->{\fig{1.0\textwidth}{figsCorona/SEIR.png}}
\vspace{-1em}

%Chained model equations for $S$, $I$ and $R$ (fraction $N_R/N$ of
%recovered/removed persons): 

\pause
\maintextbox{0.6\textwidth}{{\large
$\begin{array}{ll}
\abl{S}{t} &=-\beta I S,\\[0.5em]
\abl{E}{t} &=+\beta I S - \alpha E,\\[0.5em]
\abl{I}{t} &=+\alpha E - \gamma I,\\[0.5em]
\abl{R}{t} &=+\gamma I
\end{array}$
}
\qquad
SEIR model
}
\vspace{-1ex}

\bi
\pause \itemAsk \colAsk{Show that
  $R_0=\beta/\gamma$ and that the initial time for doubling of the
  infected is given by 
$\tau=(1/\gamma+1/\alpha)/\text{log}_2(R_0)$}\\[-1ex]
\pause \itemAnswer \colAnswer{\tiny{$R_0$ as in the SIR model. The average
    time for passing an infection is the\\[-1ex] sum $1/\gamma+1/\alpha$ of
    the incubation and 
    infection times. In this timescale, there are $\text{log}_2(R_0)$
    doublings. }}
\ei
}


}



%##############################################################
\frame{\frametitle{SIR \emph{vs.} SEIR model simulations}
%##############################################################

\placebox{0.5}{0.43}{
 \figSimple{0.95\textwidth}{figsCorona/modelSEIR.png}}
\visible<2>{\placebox{0.5}{0.43}{
 \figSimple{0.95\textwidth}{figsCorona/modelSIR.png}}}
\visible<3>{\placebox{0.5}{0.43}{
 \figSimple{0.95\textwidth}{figsCorona/modelSEIR.png}}}
}

%##############################################################
\frame{\frametitle{SEIR model with seasons (winter is ``flu'' time)}
%##############################################################

{\small
\placebox{0.5}{0.80}{\parbox{0.8\textwidth}{
\bi
\pause \item [$\Rightarrow$] make the reproduction number $R_0(t)$ time dependent
\\[-0.5ex]
\pause \item [$\Rightarrow$] %Recovery $\gamma$ and incubation $\alpha$ unchanged
infection rate
  $\beta$ variable: $\beta=\gamma R_0(t)$
\ei
}}
}

\visible<4->{\placebox{0.5}{0.37}{
 \figSimple{0.90\textwidth}{figsCorona/modelSEIRseason.png}}}

}

%##############################################################
\frame{\frametitle{Iterated map models}
%##############################################################
\bi
\item The SI, SIR, SEIR models were \bfdef{ordinary differential
  equations} (ODEs)
\pause \item Another more direct approach are \bfdef{iterated maps}:
  models for time evolution by classical model chaining
\pause \item can be interpreted as numerical solutions of ODEs but they are
  more flexible allowing ``real'' memory, e.g., truly nonzero
  incubation time instead of an exponential distributed one
\pause \item Of course, this also means we need initialize all past values
  within the memory time
\ei
}


%##############################################################
\frame{\frametitle{Iterated SIR model with memory (SIRM)}
%##############################################################

{\small
\bi
\item An infected person contacts $R_0$ persons and infects $R_0 S$
  persons \emph{exactly} 
  $\tau_I$ days after  his/her own infection
\pause \item [$\Rightarrow$] need
  history of all fractions
  $I_{t'}$ of persons infected \emph{exactly} at day $t' \le t$
\pause \item The person recovers exactly $\tau_R$ days after infection
\pause \item [$\Rightarrow$] The total fraction of ill persons (\emph{active
  cases}) at day $t$ is given by 
\bdm
I(t)=\sum_{j=i-\tau_I+1}^i I_j
\edm
\ei
}

\pause
\maintextbox{0.75\textwidth}{{\large
$\begin{array}{ll}
I_t &=R_0S(t-\tau_I)I_{t-\tau_I},\\[0.5em]
S(t) &=S(t-1)-I_t,\\
R(t) &= R(t-1) + I_{t-\tau_R},\\
I(t) &= 1-S(t)-R(t)
\end{array}$
}
\qquad
\parbox{0.20\textwidth}{SIR model\\with memory}
}
{\small Notice that the recovery does not influence the infection process
since only infection day $\tau_I$ is contagious}

}

%##############################################################
\frame{\frametitle{Simulation of the SIR model with memory}
%##############################################################

\placebox{0.5}{0.46}{
  \figSimple{0.95\textwidth}{figsCorona/modelSIRiterated.png}}

}



\subsection*{1a.2 Microscopic Models}
%##############################################################
\frame{\frametitle{1a.2 Microscopic Models}
%##############################################################
The principle is straightforward: Just break down the compartemental
models to single persons (remember the definition of a microscopic
model!)
{\small
\bi
\pause \item The health status of each person $i$ is exactly one out of a
  set, e.g. status $\in \{$ S, E, I, R $\}$
\pause \item Transition $S_i\to E_i$ if an S person $i$ is 
sufficiently close to an I person $j$ sufficiently long, e.g.
\bdm
S_i(t) \to E_i(t) \quad \text{if} \quad d_{ij}(t') \le \unit[1.5]{m}
\ \forall \ t': t-\tau_E\le t' \le t
\edm
\\[-1em]
\pause \item Transition to an I person after an incubation time
$\tau_I$
\pause \item Transition to an R person after a time period $\tau_R>\tau_I$
\ei
}

\pause So the pandemic micromodel is easy: It gets interesting when adding a
\bfdef{particle dynamics model} for the motion of the people to model,
e.g., \emph{superspreading events}


}

%##############################################################
\frame{\frametitle{Microscopic example}
%##############################################################
\visible<1>{\placebox{0.5}{0.62}{
  \figSimple{1.05\textwidth}{figsCorona/micro1.png}}}
\visible<2>{\placebox{0.5}{0.62}{
  \figSimple{1.05\textwidth}{figsCorona/micro2.png}}}
\visible<3>{\placebox{0.5}{0.62}{
  \figSimple{1.05\textwidth}{figsCorona/micro3.png}}}
\visible<4>{\placebox{0.5}{0.62}{
  \figSimple{1.05\textwidth}{figsCorona/micro4.png}}}
\visible<5>{\placebox{0.5}{0.62}{
  \figSimple{1.05\textwidth}{figsCorona/micro5.png}}}

\placebox{0.5}{0.22}{\parbox{0.9\textwidth}{
\bi
\item Time $t$: superspreading event
\pause \item Time $t+\tau$: three people infected in the middle group
\pause \item Time $t+2\tau$: one of the newly 
infected moves to the other group
\pause \item Time $t+3\tau$: incubation time over (also at the left group)
\pause \item Time $t+4\tau$: two infections in two groups
\ei
}}



}


\subsection*{1a.3 Down to Earth: Data-Related Issues}

%##############################################################
\frame{\frametitle{1a.3 Down to reality/econometrics: what can be observed?}
%##############################################################

\maintextbox{0.8\textwidth}{We want to know: \bfred{\#Infections} 
$N_I(t)=N\, I(t)$,\\ 
ideally its ``age
structure'' $I_0, I_1, ..., I_t$}
\vspace{-1em}

\pause
\maintextbox{0.8\textwidth}{We do know: \bfred{\#positive tests}
 $N_T(t)$
(``cases'') and \\ \bfred{\#Covid-19 deaths} $N_D(t)$ including the
  history $t'\le t$ 
}

{\small
Many uncertainties:
\bi
\pause \item The tests have an imperfect \bfdef{sensitivity}
$P(\text{positive}|\text{infected}) \approx \unit[99]{\%}$
\pause \item ... and an imperfect \bfdef{specifity}
$P(\text{negative}|\text{not infected}) \approx \unit[99]{\%}$
%and these figures even have large uncertainties in themselves
\pause \item Different/inconsistent definitions of a ``Covid-19 death'' event
\pause \item There is a high number of untested and potentially ill people
  $\Rightarrow$ high number of unreported cases, probably $\gg N_T$
\pause \item The fraction of reported cases depends on the number of tests
  via a monotonously increasing but otherwise unknown function
\ei
}

}



\subsection*{1a.4 Simulation at \texttt{Corona-simulation.de}}

%##############################################################
\frame{\frametitle{Corona-simulation.de (as of Oct 30, 2020)}
%##############################################################

Interactive \emph{data-driven} simulator based on an extended SIRM model
\vspace{1em}

\hspace*{-0.05\textwidth}
\figSimple{1.05\textwidth}{figsCorona/Germany_oct30_cases.png}
}

%##############################################################
\frame{\frametitle{Features I: different countries}
%##############################################################
\vspace{1em}

\hspace*{-0.05\textwidth}
\figSimple{1.05\textwidth}{figsCorona/Czechia_oct30_cases.png}
}

%##############################################################
\frame{\frametitle{Features II: different windows}
%##############################################################
\placebox{0.25}{0.66}{
 \figSimple{0.5\textwidth}{figsCorona/Germany_oct30_cum.png}}
\placebox{0.75}{0.66}{
 \figSimple{0.5\textwidth}{figsCorona/Germany_oct30_log.png}}
\placebox{0.25}{0.23}{
 \figSimple{0.5\textwidth}{figsCorona/Germany_oct30_tests.png}}
\placebox{0.75}{0.23}{
 \figSimple{0.5\textwidth}{figsCorona/Germany_oct30_rates.png}}

}

%##############################################################
\frame{\frametitle{Features III: scenario-based projections}
%##############################################################
\placebox{0.52}{0.65}{
 \figSimple{0.9\textwidth}{figsCorona/Germany_oct30_cases_prog_cropped.png}}
\placebox{0.45}{0.22}{
 \figSimple{0.9\textwidth}{figsCorona/Czechia_oct30_cases_prog_cropped.png}}
\placebox{0.33}{0.78}{Germany}
\placebox{0.25}{0.35}{Czechia}

}

%##############################################################
\frame{\frametitle{Features III: ``lockdown'' shifts ``wave''}
%##############################################################
\placebox{0.50}{0.45}{
 \figSimple{1.05\textwidth}{figsCorona/Germany_oct30_cases_progLockdown.png}}

}

%##############################################################
\frame{\frametitle{Features IV: sensitivity tests, e.g., ramping up \#tests}
%##############################################################
\placebox{0.50}{0.45}{
 \figSimple{1.05\textwidth}{figsCorona/Germany_oct30_increaseTests.png}}

}

%##############################################################
\frame{\frametitle{Summary/take-home messages }
%##############################################################
\bi
\item Only \bfred{data brings us ``down to Earth''} allowing for 
\bi
\item tests of the model quality
\item doing useful things such as projection scenarios (do not forget
  Mark Twains quote about predictions!)
\ei

\pause \item Always \bfred{check definitions of events}, e.g.,
   ``Covid-19 infection'' (including all symptom free people?) or
  ``Covid-19 death'' (including
  fatal traffic accidents of a test-positive persons?)

\pause \item \bfred{Do not confuse/mix proxies with the real
  quantities}, e.g.,
  positive tests \emph{vs.} infection events. Also check how well the
  proxy \emph{represents} the interesting quantities 
  (\#positive tests is a poor proxy for the \#infections, \#\emph{recorded}
  Covid-19 death is a much better proxy for all the Covid-19 deaths)

\pause \item \bfred{Check your sample.} Is it essentially the population or only a
  small and unknown fraction thereof?

\pause \item \bfred{Be careful with exponentially growing things}
 since small changes
  in the scenario setting can greatly influence the result

\ei
}


\end{document}













