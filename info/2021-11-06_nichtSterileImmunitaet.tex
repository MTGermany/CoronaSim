%########################################################
% skeletons fuer latex2e documente
% AUSFUEHRLICHERE Version: skeletonAusfuehrlich.tex (dec04)
%########################################################
\documentclass[12pt,a4paper]{scrartcl}  
%\documentclass[twocolumn,showpacs,preprintnumbers,amsmath,amssymb]{revtex4}
%\documentclass[preprint,showpacs,preprintnumbers,amsmath,amssymb]{revtex4}
%\documentclass[a4paper]{foils}
     % Praesentation a la "Powerpoint", vgl. ~/prosperPraesentation/          
%\documentclass[tu,colorBG,slideColor,pdf]{prosper} 

\usepackage{graphicx} % definiert \includegraphics[width=???mmm]{eps Bild}
\usepackage[dvips]{color}     %Definiert \definecolor und 
                              % red,yellow,green,blue,black,gray,white
\usepackage{lscape} % provides landscape environment fuer txt rotieren:
%\begin{landscape}.. \end{landscape}

%\usepackage{german}  %Fuck all german options do not work any more (9/2015)
                      % need defs.tex or defsSkript.tex to correct/circumvent;
                      % defsSubmit.tex does not have it (since English)
\usepackage{units} %\unit[3]{m/s^2} => 3 m/s^2 etc!
\usepackage{amsfonts,amsmath,amssymb} % crucial for correct setting of
                                % vectors etc

\usepackage{eurosym}  %Euro-Symbol: \euro{Zahl} oder euro{}

\usepackage{hyperref} 
%                  Internet-Link: \href{http://www.WasAuchImmer.html}
%                  {\blue{\underline{TextDesLinks} }}
%                  Lokaler Link: \hyperlink{targetName}{TextDesLinks}
%                  Link-Target: \hypertarget{targetName}
%                  Lokaler File-Link: \href{file:///home/ ...}
%  Info: \myHyperlink{http..}{Linktext}

\usepackage{cite} % z.B. Refs 1,2,3,5 werden zu [1-3,5]  zusammengefasst


%\input{defs}          % includes auch ``colors.st''
\input{defsSkript}          % includes auch ``colors.st''
%\input{defsSubmit}          % includes auch ``colors.st''

%\textheight230mm
%\textwidth150mm
%\oddsidemargin0mm
%\topmargin0mm
%\parindent0mm

\renewcommand{\textfraction}{0.2}  % Figures: At least 20% text at each page
\setlength{\parindent}{0.9em} %horizontal indentation first line
\setlength{\parskip}{0.5em}  %addtl. vertical space between paragraphs

\usepackage{psfrag}




%########################################################

\begin{document}
\begin{center}
\secfont{Non-sterile immunity. How to model it?}
\\[5mm]
\bfsf{Martin Treiber}
\\
\today
\end{center}


\section{Problem statement}

In Spring 2021, we thought that the new vaccinations offer a nearly
perfect protection $1-\alpha=\unit[95]{\%}$ against infection, and even
more against the more severe endpoints such as ICS or death). Moreover,
we thought that the immunity is \emph{sterile}, i.e., vaccinated people
(even the infected ones)  are not
contagious. Then, the effective reproduction number $R\subb{eff}$ in
terms of the  base reproduction number $R_0$ depends on the
vaccination protection
factor $1-\alpha$  and the percentage $p\subb{vacc}$ of fully vaccinated people via
%
\bdm
R\subb{eff}=R_0(1-p\subb{vacc}).
\edm
%
Particularly, at \unit[100]{\%} vaccination, we have
$R\subb{eff}=0$ since, per assumption, no vaccinated person can infect
others. If infected vaccinated persons \emph{can} infect, we would
have $R\subb{eff}=R_0(1-p\subb{vacc})+R_0\alpha p\subb{vacc}$ which,
for $p\subb{vacc}=1$ would be deep in the herd-immunity
range. Furthermore, if
\unit[50]{\%} were vaccinated, we would have, among the infected, an
odds ratio 
\bdm
\text{\#not vaccinated : \#vaccinated} =1/\alpha:1=20:1, 
\edm
regardless whether infected vaccinated persons are sterile or not.

However, it turned out (as of November 2021) that sterile immunity
does not apply. In fact, we have four reproduction rates:
\bi
\item $R_{00}$ for the infection path not vaccinated $\to$ not vaccinated
\item $R_{01}$ for the path not vaccinated $\to$ vaccinated
\item $R_{10}$ for the path vaccinated $\to$ not vaccinated
\item $R_{11}$ for the path vaccinated $\to$ vaccinated
\ei
Here, $R_{00}$ denotes the average number of people that an infected
not vaccinated person would infect during his/her desease if nobody were
vaccinated, while the same person would infect an averaged number
$R_{01}$ of vaccinated persons if all others were vaccinated. 

In the
old assmuption of a sterile immunity, these would be the only
reproduction numbers. If, however, the vaccinated persons are not
sterile (with 
respect to contagions), an infected vaccinated person would infect (on
average) $R_{10}$ persons if everybody else were not vaccinated, and
$R_{11}$ persons if everybody else were vaccinated.
\newpage

Hence, we have following problem statement:
\medskip

\noindent \hspace*{0.10\textwidth}
\parbox{0.8\textwidth}{
 \benum \item \textit{What is the effective reproduction rate
    $R\subb{eff}$ as a function of the four $R$ values and the
    percentage $p\subb{vacc}$ of vaccinated people?} 
\item \textit{What is the odds
    ratio \#not vaccinated : \#vaccinated among the infected people as a
    function of the $R_{ij}$ and $p\subb{vacc}$?}
\eenum}









%##########################################################


% \myHyperlink{https://www.traffic-simulation.de/index.html}
%{www.traffic-simulation.de}.

%##########################################################
% References
%##########################################################

%(1) Erstellen

% bibstyles (voller Pfad noetig!): /home/vwitme/staff/treiber/tex/inputs/*.bst
% bib-Datenbanken (voller Pfad noetig!): /home/vwitme/data/books/bibtex/bibtex/*.bib

%\bibliographystyle{/home/vwitme/staff/treiber/tex/inputs/prsty} 
\bibliographystyle{/home/vwitme/staff/treiber/tex/inputs/alphadin}

%\bibliography{refs,/home/vwitme/data/books/bibtex/bibtex/database}  %TU
%\bibliography{refs,databaseLoc}  %Notebook

%(2a) Wenn fertig:

%\include{references.tex}

%(2b) oder (besser!) direktes Einbinden .bbl File

%(3) explizit:
\begin{thebibliography}{10}
\bibitem{Einstein-SRT}
A. Einstein, \textit{Zur Elektrodynamik bewegter K\"orper}, 
Annalen der Physik \textbf{17}, 891-921 (1905).
% siehe http://de.wikipedia.org/wiki/Albert_Einstein#Werk
\bibitem{Einstein-Emc2}
A. Einstein, \textit{Ist die Tr\"agheit eines K\"orpers von seinem 
Energieinhalt abh\"angig?}, Annalen der Physik \textbf{18}, 639-641 (1905).

\end{thebibliography}

\end{document}
%########################################################

